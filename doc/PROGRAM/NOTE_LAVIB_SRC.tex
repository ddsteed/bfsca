%%%%%%%%%%%%%%%%%%%%%%%%%%%%%%%%%%%%%%%%%%%%%%%%%%%%%%%%%%%%%%%%%%%%%%%%%%%%%%%
\documentclass[aps,pra,groupedaddress,
%               showpacs,%      display the PACS code(s)
                amsfonts,amssymb,
%               twocolumn,
                preprint
    ]{revtex4}
% ---------------------- load CTAN & macro packages ---------------------------
\usepackage{amsmath}
\usepackage{revsymb}
\usepackage{natbib}
\usepackage{SE}
\usepackage{dcolumn}
%%%%%%%%%%%%%%%%%%%%%%%%%%%%%%%%%%%%%%%%%%%%%%%%%%%%%%%%%%%%%%%%%%%%%%%%%%%%%%%
\newcommand{\code}[1]{\texttt{#1}}

%%%%%%%%%%%%%%%%%%%%%%%%%%%%%%%%%%%%%%%%%%%%%%%%%%%%%%%%%%%%%%%%%%%%%%%%%%%%%%%
\begin{document}
%
%\preprint{Version 1.0: Aug 09, 2005}
\date{\today}
%
\title[LAVIB SOURCE]
{Notes on LAVIB Package Source Codes}

\author{Hao \surname{Feng}}
\email[Electronic address: ]{ddsteed@163.com}
\affiliation{College of Physics, Sichuan University, Chengdu, Sichuan,
610065, P.R.China}

\begin{abstract}
  This document is about all of the source codes of LAVIB package, which
  is modified from the original one of Prof.~Michael A.~Morrison's group
  in Oklahoma University. It is useful to note the detailed reasons
  behind the changes, so the detailed descriptions of what have been
  modified are documented here.
\end{abstract}

\maketitle
\tableofcontents
% -----------------------------------------------------------------------------

\section{Code Convention}
\label{sec:code-convention}
Some original source codes are NOT correctly indented, so I have to
format them line by line. I know that it will be difficult to compare
the original source code and the new one and find what have been
modified if every line is different. However, it is very helpful for the
programmer to read and understand the source code if good indentation
convention is made consistently. So I formatted all of the source codes
by using GNU Emacs editor and the indentation is 3 spaces except for the
classic math library function. 

I shall account for every modification for each source code in this
document. (NOTE: To generalize this package to $e$-N$_2$, some arrays
are augmented. I do not explain it one by one.)

\subsection{Data Format}
\label{sec:data-format}
In original codes, the potential data are with format of ``D23.16''. So
the local potential are like
\begin{verbatim}
    0.01000-0.1749182639341561E+00
\end{verbatim}
in which the first is the distance between the incident electron and the
target, the second is the local potential. They are too close to be read
as free format and some other software, such as xmgr, cannot read them.
So I modified the second as ``D24.16'', which added one blank between
them.

However, Andy reminded me that this package should be compatible with
other codes. If I modified the format as ``D24.16'', the other codes
will not read them correctly. So I kept all the data formats.

\section{alamn.f}
\code{alamn} is to calcuate a molecular charge distribution and expand
it in Legendre polynomials for a linear diatomic or triatomic
molecule.

\subsection{Correction to large-$r$ fit of $a_\lambda$}
In subroutine \code{ALAM}.

\subsubsection{Basic Concepts}
Suppose the charge density of a closed-shell $N$-electron linear
molecule be expanded in Legendre polynomials $P_\lambda(\cos\theta)$,
viz.,
\begin{equation}
  \label{eq:ELP}
  \rho(r,\theta) = \sum_{\lambda=0}^{\lambda_max}a_\lambda(r)
                   P_\lambda(\cos\theta)
\end{equation}
From the orthogonality relation for the Legendre polynomials, we find
that the $\lambda$th expansion coefficient in \Eq{ELP} is given as a
function of $r$ by
\begin{equation}
  \label{eq:sal}
  a_\lambda(r) = \dfrac{2\lambda+1}{2}\int_0^\pi\rho(r,\theta)
                 P_\lambda(\cos\theta)\sin\theta\,\rd\theta
\end{equation}
The integral in \Eq{sal} is evaluated for values of $r$ in a
user-prescribed $n$-point mesh \{$r_i$, $i$ = 1, 2, \ldots, $n$\} by a
32-point Gauss-Legendre quadrature.

As $r$ becomes large, the expansion coefficients $a_\lambda(r)$ of the
molecular charge density $\rho(r,\theta)$ [\Eq{sal}] can be fitted very
accurately to the analytic form
\begin{equation}
  \label{eq:lal}
  a_\lambda(r) = A_\lambda r^{p_\lambda}\exp(-\alpha_\lambda r)
\end{equation}
where $A_\lambda$, $p_\lambda$ and $\alpha_\lambda$ are real numbers.

In Ref.\cite{MAM1980}, these parameters are determined by using the
last three values of $r$ at which $a_\lambda(r)$ is evaluated from
\Eq{sal}, $r_n$, $r_{n-1}$ and $r_{n-2}$. Letting $a_n$, $a_{n-1}$ and
$a_{n-2}$ denote the values of $a_\lambda$ at these values of $r$, we
have
\begin{equation}
  \label{eq:plam}
  p_\lambda = \dfrac{h\ln\left(\dfrac{a_{n-1}}{a_n}\right) +
                      \ln\left(\dfrac{a_{n-1}}{a_{n-2}}\right)}
                    {h\ln\left(\dfrac{r_{n-1}}{r_n}\right) +
                      \ln\left(\dfrac{r_{n-1}}{r_{n-2}}\right)} 
\end{equation}
where
\begin{equation}
   h = \dfrac{r_{n-1} - r_{n-2}}{r_n - r_{n-1}}
 \end{equation}
And
\begin{align}
  \label{eq:alphlam}
  \alpha_\lambda &= \dfrac{1}{r_{n-1} - r_n}
                    \ln\left[\dfrac{a_n}{a_{n-1}}
                       \left(\dfrac{r_{n-1}}{r_n}\right)^{p_\lambda}\right] \\
  \label{eq:Alam}
  A_\lambda & = a_nr_n^{-p_\lambda}\exp(\alpha_\lambda r_n)
\end{align}
NOTICE: There is a print error in Eq.~(16) in Ref.\cite{MAM1980},
the source code of ALAMN is consistent with \Eq{Alam}.

Thus the expansion coefficients and hence the charge density need not be
calculated for $r > r_n$ even if $a_\lambda(r)$ is required for large-$r$
values of $r$.


\subsubsection{Saha's correction to $e$-N$_2$}
We can get well-behaved large-$r$ $a_\lambda$ and static potential
$V_{st}$ for $e$-H$_2$ by using \Eq{lal}, \Eq{plam}, \Eq{alphlam} and
\Eq{Alam}. But for $e$-N$_2$, if we choose the $r$-mesh as
\begin{center}
\newcolumntype{.}{D{.}{.}{-1}}
\begin{tabular}{...}
  0.00 &  0.01  &   1.20  \\
  1.20 &  0.02  &   2.00  \\
  2.00 &  0.04  &   4.40  \\
  4.40 &  0.08  &   6.00  
\end{tabular}
\end{center}
and choose LAMMAX = 14, we get $A_{14} = 3.093\times 10^{18}$, $p_{14} =
-65.58$ and $\alpha_{14} = -8.408$ for $R = 2.020 a_0$. Meanwhile, the
static potential ($v_\lambda$) for $\lambda = 14$ are
\begin{center}
% \newcolumntype{.}{D{.}{.}{-1}}
\begin{tabular}{ccc}
     $r$   &    $v_\lambda$ \\
   \ldots  &  \ldots                 \\
   5.04000 &  \quad  -0.1131232954588831E+93 \\
   5.12000 &  \quad  -0.1410275599185811E+93 \\
   5.20000 &  \quad  -0.1752150266189247E+93 \\
   5.28000 &  \quad  -0.2169698823790784E+93 \\
   5.36000 &  \quad  -0.2678129626035787E+93 \\
   5.44000 &  \quad  -0.3295407561285985E+93 \\
   5.52000 &  \quad  -0.4042701040798804E+93 \\
   5.60000 &  \quad  -0.4944893200460516E+93 \\
   5.68000 &  \quad  -0.6031165389052729E+93 \\
   5.76000 &  \quad  -0.7335661891117883E+93 \\
   5.84000 &  \quad  -0.8898245787188291E+93 \\
   5.92000 &  \quad  -0.1076535689486381E+94 \\
   6.00000 &  \quad -0.1299098386733144E+94
\end{tabular}
\end{center}
They are overflow! (They are too LARGE for other high $\lambda's$ and
other internuclear distances!)

I searched our store directory and found that B.~C.~Saha wrote the
following codes in the subroutine \code{ALAM}
\begin{center}
\begin{verbatim}
... ... ...
C     PSM=(DLOG(ABY2/ABY1)+RAT*DLOG(ABY2/ABY3))/(DLOG(R2/R1)+RAT*DLOG(R2
C    #/R3))
c---
      PSM=0.D+00
C     IF(IFIX.EQ.1.AND.PSM.GT.1.D+00) PSM=1.D+00
      QSM = -DLOG((ABY3/ABY2)*(R2/R3)**PSM)/(R3-R2)
C See the note for this modification 8/23/85
      ASM = Y3*R3**(-PSM)*DEXP(QSM*R3)
\end{verbatim}
\end{center}
We don't have his note, but from his codes I can guess his formula for
large-$r$ $a_\lambda$ is
\begin{equation}
  \label{eq:sahalal}
  a_\lambda(r) = A_\lambda\exp(-\alpha_\lambda r)
\end{equation}

By using Saha's formula we get $A_{14} = 1.351\times 10^{-3}$, $p_{14} =
0.0$ and $\alpha_{14} = 2.595$. The static potential ($v_\lambda$)for
$\lambda = 14$ are
\begin{center}
% \newcolumntype{.}{D{.}{.}{-1}}
\begin{tabular}{ccc}
     $r$   &    $v_\lambda$ \\
   \ldots  &  \ldots               \\
   5.04000 &  \quad  .2155357141168185E-08 \\
   5.12000 &  \quad  .1772666466746914E-08 \\
   5.20000 &  \quad  .1458346809604667E-08 \\
   5.28000 &  \quad  .1200442725400119E-08 \\
   5.36000 &  \quad  .9890022814558143E-09 \\
   5.44000 &  \quad  .8157512135613536E-09 \\
   5.52000 &  \quad  .6738244745961844E-09 \\
   5.60000 &  \quad  .5575415032036261E-09 \\
   5.68000 &  \quad  .4622160475180030E-09 \\
   5.76000 &  \quad  .3839942571407144E-09 \\
   5.84000 &  \quad  .3197165625943484E-09 \\
   5.92000 &  \quad  .2667999764303344E-09 \\
   6.00000 &  \quad  .2231387506188567E-09
\end{tabular}
\end{center}


\subsubsection{New correction to large-$r$ fit}
Although Saha's codes corrected the large-$r$ fit overflow, we can't
reproduce $e$-H$_2$ static potential because \Eq{sahalal} is different
from \Eq{lal}. I guess that for some $e$-targets, the parameters of
$A_\lambda$, $p_\lambda$ and $\alpha_\lambda$ cannot be chosen ONLY by
the last three $a_\lambda$. So I will fit them by using more
$a_\lambda's$.

\Eq{lal} can be rewritten as
\begin{equation}
  \label{eq:lalog}
  \ln a_\lambda = \ln A_\lambda + p_\lambda\ln r - \alpha_\lambda r
\end{equation}
Letting 
\begin{subequations}
  \begin{align}
    \label{eq:lalp}
    b_\lambda & = \ln a_\lambda \\
    g_{0\lambda} & = \ln A_\lambda \\
    g_{1\lambda} & = -\alpha_\lambda \\
    g_{2\lambda} & = p_\lambda
  \end{align}
\end{subequations}
Then
\begin{equation}
  \label{eq:lallog2}
  b_\lambda = g_{0\lambda} + g_{1\lambda}r + g_{2\lambda}\ln r
\end{equation}
So $b_\lambda$ is linearlly dependent on the fitting coefficients. We
adopt ``General Linear Least Squares'' routine --- SVDFIT (Signular
Value Decomposition fit) \cite{PTV1992} to fit the coefficients.

By using 10 $a_\lambda$ to fit \Eq{lallog2} we get $A_{14} = 2.597\times
10^{-4}$, $p_{14} = 6.346$ and $\alpha_{14} = 4.627$. The static
potential ($v_\lambda$) for $\lambda = 14$ are
\begin{center}
% \newcolumntype{.}{D{.}{.}{-1}}
\begin{tabular}{ccc}
     $r$   &    $v_\lambda$ \\
   \ldots  &  \ldots               \\
   5.04000 &   \quad 0.2154651146898829E-08 \\
   5.12000 &   \quad 0.1771786323990111E-08 \\
   5.20000 &   \quad 0.1457253305351663E-08 \\
   5.28000 &   \quad 0.1199088632125510E-08 \\
   5.36000 &   \quad 0.9873308802109118E-09 \\
   5.44000 &   \quad 0.8136945736411463E-09 \\
   5.52000 &   \quad 0.6713014542439685E-09 \\
   5.60000 &   \quad 0.5544554312783496E-09 \\
   5.68000 &   \quad 0.4584520410108206E-09 \\
   5.76000 &   \quad 0.3794161238428322E-09 \\
   5.84000 &   \quad 0.3141632320379888E-09 \\
   5.92000 &   \quad 0.2600813953949083E-09 \\
   6.00000 &   \quad 0.2150311719575100E-09
\end{tabular}
\end{center}

By using 80 $a_\lambda$ to fit \Eq{lallog2} we get $A_{14} = 25.275$,
$p_{14} = -10.067$ and $\alpha_{14} = 1.217$. The static potential for
$\lambda = 14$ are
\begin{center}
% \newcolumntype{.}{D{.}{.}{-1}}
\begin{tabular}{ccc}
     $r$   &    $v_\lambda$ \\
   \ldots  &  \ldots               \\
   5.04000 &   \quad 0.2155529635889707E-08 \\
   5.12000 &   \quad 0.1772881510956653E-08 \\
   5.20000 &   \quad 0.1458613984169379E-08 \\
   5.28000 &   \quad 0.1200773569363670E-08 \\
   5.36000 &   \quad 0.9894106528899811E-09 \\
   5.44000 &   \quad 0.8162537098927050E-09 \\
   5.52000 &   \quad 0.6744409210914880E-09 \\
   5.60000 &   \quad 0.5582955194052478E-09 \\
   5.68000 &   \quad 0.4631357026518250E-09 \\
   5.76000 &   \quad 0.3851128268835364E-09 \\
   5.84000 &   \quad 0.3210734010977414E-09 \\
   5.92000 &   \quad 0.2684415191666285E-09 \\
   6.00000 &   \quad 0.2251196654564229E-09
\end{tabular}
\end{center}

I still keep the linear equation method (LEM) to solve $A_\lambda$,
$p_\lambda$ and $\alpha_\lambda$ by solving \Eq{plam}, \Eq{alphlam} and
\Eq{Alam}. So for H$_2$, we have two methods to calculate large-$r$
$a_\lambda$ and static potential ($v_\lambda$). For $\lambda=6$, we get
\begin{center}
%\newcolumntype{.}{D{.}{.}{-1}}
\begin{tabular}{lcc}
                 & LEM & SVDFIT \\
     $A_\lambda$ & \quad 0.1601316757808016E-23  & \quad 0.1189776202324306E-04 \\
     $p_\lambda$ & \quad 0.4523276259430859E+02  & \quad 0.6724227178999224E+01 \\
$\alpha_\lambda$ & \quad 0.7923329439442724E+01  & \quad 0.3361975965910192E+01
\end{tabular}
\end{center}
and static potential ($v_\lambda$)
\begin{center}
%\newcolumntype{.}{D{.}{.}{-1}}
\begin{tabular}{rcc}
                     & $v_\lambda$ (LEM) & $v_\lambda$ (SVDFIT) \\
   \ldots  &  \ldots      & \ldots         \\
   8.90000 &  \quad -0.3738601412980983E-07 &   \quad  -0.3738604465762323E-07  \\
   9.00000 &  \quad -0.3457238352675295E-07 &   \quad  -0.3457241617130511E-07  \\
   9.10000 &  \quad -0.3199833271303255E-07 &   \quad  -0.3199836759524417E-07  \\
   9.20000 &  \quad -0.2964111789027305E-07 &   \quad  -0.2964115513652894E-07  \\
   9.30000 &  \quad -0.2748037669302244E-07 &   \quad  -0.2748041643535470E-07  \\
   9.40000 &  \quad -0.2549785533465599E-07 &   \quad  -0.2549789771093117E-07  \\
   9.50000 &  \quad -0.2367716973439954E-07 &   \quad  -0.2367721488850992E-07  \\
   9.60000 &  \quad -0.2200359606053337E-07 &   \quad  -0.2200364414259235E-07  \\
   9.70000 &  \quad -0.2046388677921551E-07 &   \quad  -0.2046393794575718E-07  \\
   9.80000 &  \quad -0.1904610885409454E-07 &   \quad  -0.1904616326827752E-07  \\
   9.90000 &  \quad -0.1773950121435192E-07 &   \quad  -0.1773955904616741E-07  \\
  10.00000 &  \quad -0.1653434900198538E-07 &   \quad  -0.1653441042846955E-07  

\end{tabular}
\end{center}

\subsection{Fix the bug of calculating the harmonic coefficient of $\pi$-orbital}
\label{sec:fix-bug-calculating}
In subroutine \code{SPHPRJ}, when the $\pi$-orbital is calculated, the
original code has a minor bug. The original code is
\begin{verbatim}
DO MOIND = 1, NMO
... ... ...
if (M .EQ. 2) GO TO 101
ENDDO
101 continue
\end{verbatim}
$M = 2$ means the orbital is $\pi_y$. In this code, the $xz$-plane can
be chosen so that $\pi_y$ does not contribute. So we skip $\pi_y$
orbital. However, N$_2$ has 7 bound orbitals which include one $\pi_x$
and one $\pi_y$. If the $\pi_x$ is before $\pi_y$, such as for $R=2.020
a_0$, the original code is right. But if the $\pi_y$ is before the
$\pi_x$, like $R=1.700 a_0$, the original code will NOT calculate the
$a_\lambda$ of $\pi_x$. To fix the bug, there's a simple way,
\begin{verbatim}
DO MOIND = 1, NMO
... ... ...
if (M .EQ. 2) GO TO 101
101 continue
ENDDO
\end{verbatim}
i.e., for $\pi_y$ orbital, we do not jump out of the do loop. So whether
$\pi_y$ is before $\pi_x$ or not, we can get the correct $a_\lambda$ of
$\pi$-orbital. 

However, someone else has calculated $e$-N$_2$ by using \code{alamn}.
Why is there still a bug? I think that it would be right if all
\code{ENDDO} are replaced by \code{CONTINUE}. i.e., 
\begin{verbatim}
DO 110 MOIND = 1, NMO
... ... ...
if (M .EQ. 2) GO TO 101
101 continue
\end{verbatim}
the loop would exit only when \code{MOIND} = \code{NMO}.

\subsection{Input of GTO Basis}
\label{sec:input-gto-basis}
In the original codes, the GTO basis sets are generated by POLYATOM
package. The basis sets are like
\begin{verbatim}
3 1 1 1 1 1 1 1 1 3 1 1 1 1 1 1 1 1                    
H1        S           33.644400   0.0253740             
H1        S           5.0579600   0.1896830             
H1        S           1.1468000   0.8529300             
H1        S           0.3211440   1.0000000             
H1        S           0.1013090   1.0000000             
H1        X           2.2280000   1.0000000             
H1        X           0.5183000   1.0000000             
H1        Y           2.2280000   1.0000000             
H1        Y           0.5183000   1.0000000             
H1        Z           2.2280000   1.0000000             
H1        Z           0.5183000   1.0000000             
H2        S           33.644400   0.0253740             
H2        S           5.0579600   0.1896830             
H2        S           1.1468000   0.8529300             
H2        S           0.3211440   1.0000000             
H2        S           0.1013090   1.0000000             
H2        X           2.2280000   1.0000000             
H2        X           0.5183000   1.0000000             
H2        Y           2.2280000   1.0000000             
H2        Y           0.5183000   1.0000000             
H2        Z           2.2280000   1.0000000             
H2        Z           0.5183000   1.0000000
\end{verbatim}
However, we are currently using GAMESS package to calculate the bound
orbitals and the electronic density of the target. Moreover, the
polarization potential (BTAD/DSG) are also calculated by the modified
GAMESS. So I modified the format of \code{ITYPE} considering the output
of GAMESS, (for example, for N$_2$ the basis sets are like)
\begin{verbatim}
 4 2 1 1 1 1 3 3 3 1 1 1 1 1 1 1 1 1 1 1 1 1 1 1 1 1 1 1 1 1
 4 2 1 1 1 1 3 3 3 1 1 1 1 1 1 1 1 1 1 1 1 1 1 1 1 1 1 1 1 1
 1     S       5909.440000       .006240
 1     S        887.451000       .047669
 1     S        204.749000       .231317
 1     S         59.837600       .788869
 1     S         19.998100       .792912
 1     S          2.686000       .323609
 1     S          7.192700      1.000000
 1     S           .700000      1.000000
 1     S           .213300      1.000000
 1     S           .060000      1.000000
 1     X         26.786000       .038244
 1     X          5.956400       .243846
 1     X          1.707400       .817193
 1     Y         26.786000       .038244
 1     Y          5.956400       .243846
 1     Y          1.707400       .817193
 1     Z         26.786000       .038244
 1     Z          5.956400       .243846
 1     Z          1.707400       .817193
 1     X           .531400      1.000000
 1     Y           .531400      1.000000
 1     Z           .531400      1.000000
 1     X           .165400      1.000000
 1     Y           .165400      1.000000
 1     Z           .165400      1.000000
 1     X           .050000      1.000000
 1     Y           .050000      1.000000
 1     Z           .050000      1.000000
 1    XX           .980000      1.000000
 1    YY           .980000      1.000000
 1    ZZ           .980000      1.000000
 1    XY           .980000      1.000000
 1    XZ           .980000      1.000000
 1    YZ           .980000      1.000000
 1    XX           .160000      1.000000
 1    YY           .160000      1.000000
 1    ZZ           .160000      1.000000
 1    XY           .160000      1.000000
 1    XZ           .160000      1.000000
 1    YZ           .160000      1.000000
 2     S       5909.440000       .006240
 2     S        887.451000       .047669
 2     S        204.749000       .231317
 2     S         59.837600       .788869
 2     S         19.998100       .792912
 2     S          2.686000       .323609
 2     S          7.192700      1.000000
 2     S           .700000      1.000000
 2     S           .213300      1.000000
 2     S           .060000      1.000000
 2     X         26.786000       .038244
 2     X          5.956400       .243846
 2     X          1.707400       .817193
 2     Y         26.786000       .038244
 2     Y          5.956400       .243846
 2     Y          1.707400       .817193
 2     Z         26.786000       .038244
 2     Z          5.956400       .243846
 2     Z          1.707400       .817193
 2     X           .531400      1.000000
 2     Y           .531400      1.000000
 2     Z           .531400      1.000000
 2     X           .165400      1.000000
 2     Y           .165400      1.000000
 2     Z           .165400      1.000000
 2     X           .050000      1.000000
 2     Y           .050000      1.000000
 2     Z           .050000      1.000000
 2    XX           .980000      1.000000
 2    YY           .980000      1.000000
 2    ZZ           .980000      1.000000
 2    XY           .980000      1.000000
 2    XZ           .980000      1.000000
 2    YZ           .980000      1.000000
 2    XX           .160000      1.000000
 2    YY           .160000      1.000000
 2    ZZ           .160000      1.000000
 2    XY           .160000      1.000000
 2    XZ           .160000      1.000000
 2    YZ           .160000      1.000000
\end{verbatim}
So, the \code{ITYPE} are modified from
\begin{verbatim}
 DATA  ITYPE/'S  ','X  ','Y  ','Z  ','XX ','YY ','ZZ ','XY ','XZ ',
X'YZ ','XXX','YYY','ZZZ','XXY','XXZ','XYY','YYZ','XZZ','YZZ','XYZ'/
\end{verbatim}
to
\begin{verbatim}
 DATA  ITYPE/'  S','  X','  Y','  Z',' XX',' YY',' ZZ',' XY',' XZ',
$     ' YZ','XXX','YYY','ZZZ','XXY','XXZ','XYY','YYZ','XZZ','YZZ',
$     'XYZ'/
\end{verbatim}

\section{vlam.f}

\section{vibker.f}
\label{sec:vibker.f}
\code{vibker} is to calculate the exchange kernels at a given
internuclear geometries so they may be used in the construction of a
vibrational exchange kernel. Now, it is ONLY for \textit{homonuclear}
diatomic molecule.

\subsection{Minor Modifications}
\label{sec:minor-modifications}
\begin{itemize}
\item In Original codes, the names of input/output files are read
  directly through the shell script. In subroutine \code{indata}
\begin{verbatim}
       read(5,*)sphnam
       indsph=index(sphnam,' ')
       read(5,*)kernfil
       indker=index(kernfil,' ')
       read(5,*)(geomnam(i),i=1,ngeom)
       write(6,*)
       write(6,*)' The kernel filenames (symmetry label will be added):'
       do i=1,ngeom
         indgeo(i)=index(geomnam(i),' ')
         if(indgeo(i).eq.0) indgeo(i)=9
         write(6,*)sphnam(1:indsph-1)//geomnam(i)(1:indgeo(i)-1)//dat,
     +   ' ',kernfil(1:indker-1)//geomnam(i)(1:indgeo(i)-1)//dat
       end do
\end{verbatim}
and in the \code{main} function,
\begin{verbatim}
       open(unit=10,file=sphnam(1:indsph-1)//
      +       geomnam(igeom)(1:indgeo(igeom)-1)//dat,status='old',
      +       form='unformatted')
       ... ... ...
       open(unit=11,file=kernlfil(1:indker-1)//
      +        geomnam(igeom)(1:indgeo(igeom)-1)//
      +        symlab(isymind)(1:2)//ker,
      +        form='unformatted')
\end{verbatim}
  For example, \code{sphnam} is ``h2sphrj'', \code{kernfil} is ``h2''
  and \code{geomnam} is ``r14''. However, it is not convenient since
  every name should be less than 8 characters---it is often forgotten.
  So I copy the input file as \code{fort.*} and read the data through
  the corresponding channel in the code.
\item In original codes, all the geometries to be calculated are read as
  an array \code{geomnam}. I remove the number of geometries and
  calculate one given geometry once so that I can loop different
  geometry in the shell script --- it is clearer and more understandable.
\end{itemize}

\subsection{$\pi$-orbital}
The original code could only deal with $\sigma$-orbital, so we have to augment it to include $\pi$-orbital for $e$-N$_2$ scattering. The general exchange kernel is
\begin{equation}
  \label{eq:KEXb}
  \vckexk = \sum_{i=1}^{N_{occ}}\sum_{\ell''\ell'''}
  \orwbcn\orwcbn\sum_{\lambda}\glmbn\dfrac{r_<{}^\lambda}{r_>{}^{\lambda+1}}
\end{equation}
with
\begin{equation}
  \label{eq:glmbn}
  \begin{split}
  \glmbn & = \sqrt{\dfrac{(2\ell+1)(2\ell'+1)}{(2\ell''+1)(2\ell'''+1)}}\,
  \CG{\ell\lambda\ell'';-\Lambda,\Lambda-m_i}\,\CG{\ell\lambda\ell'';00} \\
  & \otimes \CG{\ell'\lambda\ell''';\Lambda,m_i-\Lambda}\,\CG{\ell'\lambda\ell''';00}
  \end{split}
\end{equation}

If the isolated molecule only contains $\sigma$ and $\pi$ orbitals, both
of which are fully occupied, we have,
\begin{equation}
  \label{eq:KEXc}
  \begin{split}
  \vckexk & = \sum_{i=1}^{N_{occ}(\sigma)}\sum_{\ell''\ell'''\lambda}
  \orwbcsn\orwcbsn\dfrac{r_<{}^\lambda}{r_>{}^{\lambda+1}}\Ala \\
       & + \sum_{i=1}^{N_{occ}(\pi)}\sum_{\ell''\ell'''\lambda}
  \orwbcpn\orwcbpn\dfrac{r_<{}^\lambda}{r_>{}^{\lambda+1}}\Bla
  \end{split}
\end{equation}
where
\begin{equation}
  \label{eq:Ala}
  \Ala = \glas
\end{equation}
and 
\begin{equation}
  \label{eq:Bla}
  \Bla = \glabpa + \glabpb
\end{equation}
We must note that in \Eq{KEXc} \textsl{there is only one $\pi$ orbital
  for N$_2$} since $\pi(x)$ and $\pi(y)$ are summed up through \Eq{Bla}.

For $e$-H$_2$, we only calculate $\Ala$ and $N_{occ}=1$.  However, for $e$-N$_2$, both $\Ala$ and $\Bla$ have to be calculated where $N_{occ}(\sigma)=5$ and $N_{occ}(\pi)=1$. To do this, another two arrays \code{clebx4(nbound,nexdim,nlamex,nexdim)} and \code{clebx5(nbound,nexdim,nlamex,nexdim)} are defined.

\subsubsection{clebexch}
In subroutine \code{clebexch}, the Clebsch-Gordon elements are calculated. From \Eq{glmbn}, three C-G coefficients have to be calculated,
\begin{itemize}
  \item $\CG{\ell\lambda\ell'';-\Lambda,\Lambda-m_i,-m_i}$
  \item $\CG{\ell\lambda\ell'';0,0,0}$
  \item $\CG{\ell\lambda\ell'';\Lambda,m_i-\Lambda,m_i}$
\end{itemize}
 
\section{lavib.f}

\bibliography{SCIENCE_REFERENCES}

\end{document}
