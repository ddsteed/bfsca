%%%%%%%%%%%%%%%%%%%%%%%%%%%%%%%%%%%%%%%%%%%%%%%%%%%%%%%%%%%%%%%%%%%%%%%%%%%%%%%
\documentclass[aps,pra,groupedaddress,
%               showpacs,%      display the PACS code(s)
                amsfonts,amssymb,
%               twocolumn,
                preprint
    ]{revtex4}
% ---------------------- load CTAN & macro packages ---------------------------
\usepackage{amsmath}
\usepackage{revsymb}
\usepackage{natbib}
%\usepackage{SE}
\usepackage{dcolumn}
%%%%%%%%%%%%%%%%%%%%%%%%%%%%%%%%%%%%%%%%%%%%%%%%%%%%%%%%%%%%%%%%%%%%%%%%%%%%%%%
\newcommand{\molnwv}{\psi}
\newcommand{\vibqu}{\upsilon}

%%%%%%%%%%%%%%%%%%%%%%%%%%%%%%%%%%%%%%%%%%%%%%%%%%%%%%%%%%%%%%%%%%%%%%%%%%%%%%%
\begin{document}
%
%\preprint{Version 1.0: Oct 03, 2005}
\date{\today}
%
\title[WLAM: Calculation of the coupled potential]
{WLAM: Calculation of the Coupled Potential}

\author{Hao \surname{Feng}}
\email[Electronic address: ]{ddsteed@163.com}
\affiliation{College of Physics, Sichuan University, Chengdu, Sichuan,
610065, P.R.China}

\maketitle

\section{Coupled Vibrational Potentials}
Both coulped vibrational potentials and multi-moments, including
spherical and non-spherical polarizabilited and qudrupole, etc., can be
expressed as
\begin{equation}  \label{equ:cci}
I = \int_0^\infty \molnwv_\vibqu^*(R)f(R)\molnwv_{\vibqu'}(R)\,dR
\end{equation}
where~$f(R)$~denotes the potentials or multi-moments, which is varied as
the internuclear distance $R$. $\molnwv$~are the molecular vibrational
wavefunctions. Since $R > 0$, both $\molnwv$ and $f(R)$ do not exist as
$R < 0$, and we can take them as zero. So, Eq.~(\ref{equ:cci}) can be
written as
\begin{equation}  \label{equ:ccii}
I = \int_{-\infty}^\infty \molnwv_\vibqu^*(R)f(R)\molnwv_{\vibqu'}(R)\,dR
\end{equation}

Eq.~(\ref{equ:ccii}) can be solved by using Gauss-Hermite quadratures.
However, Gauss-Hermite quadratures are for the whole ($-\infty \sim
\infty$) range, while we can get limited $f(R)$. In order to use
Gauss-Hermite quadratures, we have to suppose that $f(R)$ is smooth
within the whole range.

Suppose we have $N$ Gauss-Hermite quadratures: $\xi_1$, $\xi_2$, \dots,
$\xi_{N}$,Eq.~(\ref{equ:ccii})~should be
\begin{equation}  \label{equ:gauss-hermite}
I = \sum_{k=1}^{N}\lambda_k g(\xi_k)\exp\left({\xi_k}^2\right)
\end{equation}
where $\lambda_k$ is the weighted factor\footnote{Both $\xi_k$ and
  $\lambda_k$ can be got through mathematical manual. Notice that
  $\lambda_k$ are different for different $N$ and $\xi_k$.}. $g(R)$ is
the integrand.
\begin{equation}
g(R)=\molnwv_\vibqu^*(R)f(R)\molnwv_{\vibqu'}(R)
\end{equation}

If $g(R)$ are got only within $[R_1, R_M]$, we have to map Gauss-Hermite
quadratures into this range. Suppose
\begin{equation}
R = R_\alpha\times\left(\frac{X}{\alpha}+R_x\right)\times R_e
\end{equation}
where $\alpha$ and $R_e$ are constants, while $R_\alpha$ and $R_x$ are
transforming coefficients to ensure $R$ on every quadrature $X$ is within
$[R_1, R_M]$. So
\begin{equation}
\begin{split}
I &=
\int_{-\infty}^{\infty}g\left[R_\alpha\times\left(\frac{X}{\alpha}+R_x\right)\times
  R_e\right]\,d\left[R_\alpha\times\left(\frac{X}{\alpha}+R_x\right)\times R_e\right] \\
&= \frac{R_\alpha\times
  R_e}{\alpha}\int_{-\infty}^{\infty}g\left[R_\alpha\times\left(\frac{X}{\alpha}+R_x\right)\times R_e\right]\,dX 
\end{split}
\end{equation}

Notice that the integral varialbe of $R$ is turned to $X$. By using
Eq.(\ref{equ:gauss-hermite}), we get,
\begin{equation}
I = \frac{R_\alpha\times R_e}{\alpha}\sum_{k=1}^{N}\lambda_k g\left[R_\alpha\times\left(\frac{\xi_k}{\alpha}+R_x\right)\times R_e\right]\exp\left({\xi_k}^2\right)
\end{equation}

For simplicity, we let,
\begin{gather}
f_a = \frac{R_\alpha\times R_e}{\alpha} \\
\zeta_k = R_\alpha\times\left(\frac{\xi_k}{\alpha}+R_x\right)\times R_e
\end{gather}
the integral is then written as,
\begin{equation}  \label{equ:cciii}
\begin{split}
I &=
f_a\times\sum_{k=1}^{N}\lambda_kg(\zeta_k)\exp\left({\xi_k}^2\right)
\\
&=
f_a\times\sum_{k=1}^{N}\lambda_k\molnwv_{\vibqu}(\zeta_k)f(\zeta_k)\molnwv_{\vibqu'}(\zeta_k) \\
&\quad\times\exp\left[\left(\alpha\times\frac{\zeta_k-R_\alpha\times
    R_e\times R_x}{R_\alpha\times R_e}\right)^2\right]
\end{split}
\end{equation}

So, if we choose suitalbe transforming coefficients $R_\alpha$ and $R_x$
to ensure every $\zeta_k$ in Eq.(\ref{equ:cciii}) is within $[R_1,R_M]$,
we can interpolate every $\molnwv_{\vibqu}(\zeta_k)$,
$\molnwv_{\vibqu'}(\zeta_k)$ and $f(\zeta_k)$.

If we choose another transforming formula, i.e.,
\begin{gather}
\alpha = R_\alpha\times \sqrt{\omega_e\times m_u}\times R_e \\
R = \left(\frac{X}{\alpha}+R_x\right)\times R_e
\end{gather}
where $R_\alpha$, $\omega_e$, $m_u$, $R_e$ and $R_x$ are all constants.
The integral is turned
\begin{equation}
\begin{split}
I &=\int_{-\infty}^{\infty}g\left[\left(\frac{X}{\alpha}+R_x\right)\times
  R_e\right]\,d\left[\left(\frac{X}{\alpha}+R_x\right)\times R_e\right] \\
&=\frac{R_e}{\alpha}\int_{-\infty}^{\infty}g\left[\left(\frac{X}{\alpha}+R_x\right)\times
  R_e\right]\,dX
\end{split}
\end{equation}
letting
\begin{gather}
f_a = \frac{R_e}{\alpha} \\
\zeta_k = \left(\frac{\xi_k}{\alpha}+R_x\right)\times R_e
\end{gather}
then the integral is
\begin{equation}
\begin{split}
I &=
f_a\times\sum_{k=1}^{N}\lambda_kg(\zeta_k)\exp\left({\xi_k}^2\right)
\\
&=
f_a\times\sum_{k=1}^{N}\lambda_k\molnwv_{\vibqu}(\zeta_k)f(\zeta_k)\molnwv_{\vibqu'}(\zeta_k) \\
&\quad\times\exp\left[\left(\frac{\zeta_k}{R_e}-R_x\right)\times \alpha\right]^2
\end{split}
\end{equation}

\end{document}
